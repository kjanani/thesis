\section{introduction}
Social media has changed our way or looking at information. From simple consumers we have been given the opportunity to produce and act on what is potentially trending now. Each of us can  inform, comment,m and participate in his own way to the constant flow of events that apparently never ends. 

As a matter of fact, the overload of information caused by the always growing flow of stories have been subject of an uncountable number of studies \cite{}. Among which some have tried to leverage and select information adapted to the final end user \cite{}. This is particularly the case for the online media industry who have been facing many new challenges with the shift from the printed to online press. One challenge of interest is the ability to discover and track along time news events. For instance, both media internet actors, Yahoo and Twitter, propose  on their homepage a \emph{``trending now"} module aiming at helping the user to focus on the most dominant topic in the big flow of information.

Topic detection and tracking (TDT) is a subject of large attention since more than a decade \cite{} but have gained recently new interest with the advent of the social media.  TDT aims at detecting along time the main trends in flow of information and at giving to each of them a short description. By doing so, media actors can help the end user in following how topics emerge, evolve and fade. By connecting topics along time the end user is provided with a map helping him to deal with information overload.

The most effective models developed by the TDT community consists of an extension of  topic models where the time information is taken into account. This is the case with  recently introduced  NMF based models --- used traditionally for topic modeling --- by connecting along time the learned representations as the flow of document is produced \cite{}. In the same spirit, other works extend LDA for analyzing the evolution of topics along time \cite{}. However, these last approaches by relying on full bayesian modeling are  too slow to be used in practice \cite{}.


All TDT approaches  still consider to observe as input  only a flow of raw textual information. Actually, there are blind to the fact that  nowadays  information on the social media comes with metadata giving it a social context. In this work, we argue that by leveraging this social information along with the content one can track topic evolution along time in a more accurate way. For instance, it make sense to believe that when  \emph{lionel messi or christiano ronaldo} is mentioned in a tweet it is more likely that this is a tweet about football than politic. While when \emph{Obama} tweets about \emph{employment growth} it is more likely to be about politic. 

To the best of our knowledge, while using social metadata has been considered for the task of topic modeling \cite{}, it has never been considered for tracking topics along time. Therefore, we propose a novel topic tracking model to be used in a streaming environement to discover and track topic along time. Our model relies on the idea of collective matrix factorization (CMF)  by learning shared representations from both the tweet content information and from the social media information coming along with the tweet (i.e. the author and the mentions). By doing so, we learn in the same time the topics but also the communities (i.e. group of users) supporting each of the discovered topic. Our model, inspired from \cite{},  gives the possibility to connect topics and communities along time. The model is designed such that we  can set a joint or a distinct evolution pattern for topics and communities. We show that depending on the nature of the topic some exhibits a common evolution pattern with their supporting communities,  while others exhibits unrelated patterns.   We test to which extent  by using the social media context of the tweet discover better topics. We show that this is particularly the case  for \#tags that exhibit a big variation in their topics while being stable in terms of community (i.e. users engaged with this #tag).





