%!TEX root = paper.tex
There are two families of work which we delve into to provide an overview of
related papers: one is the family of work on TDE, and the other is
family of work that uses some type of link structure (either derived from citation networks,
or other means) for topic modeling.  Works which fall under the latter
family tree generally do not model the evolution of topics that are discovered, and hence do not
incorporate a temporal aspect to the model they develop.  To the best of our knowledge, 
our work is the first to combine both topic discovery and evolution with link structure.
More importantly, our work is the only one which studies where the soft spot really lies.
Meaning, we comprehensively study through experiments for what kind of topics does
the social context of an article through user interactions really produce improvements in performance.

\emph{Topic discovery and evolution} has been a subject which has garnered plenty of attention for more than a decade 
but has gained renewed interest in recent 
years with the advent of the social media \cite{Sayyadi:2009,Becker:2009}.   
The most effective models developed by the topic tracking community is generally built on
some well-known topic discovery model (or topic model) with a temporal aspect added to it to accommodate for the incoming
stream of data.  This is the case with NMF (non-negative matrix factorization) \cite{Lee:Nature}
based models that connects along time the learned representations for the incoming stream of
data \cite{Blei:2006, mairal2010, Saha:2012, Vaca:2014}. In the same spirit, other works extend 
generative models like latent dirichlet allocation (LDA) \cite{blei2003}
for analyzing the evolution of topics along time \cite{AlSumait:2008, Wang:2012, Kawamae:2011, Wang:2006}. 

\emph{Social information for topic detection} has masqueraded with many names in literature.
There have been entire lines of works which use the link structure between documents to model
topics.  This link structure can be built to model a certain relationship between documents.  Examples of
information that can be modeled through link structure are common authors between documents, citation networks, etc.
Many of these models derive inspiration from classical topic modeling algorithms, and extend them
to incorporate for the new modality of information now available to them.
\cite{Erosheva:2004} proposed the Link-LDA model which extends LDA to include citation information.
It replicates the graphical model used for modeling documents and words to also model documents and
citations.  It enforces that the document's topic distribution and the document's citation distribution
to be the same.  \cite{Nallapati:2008} propose Link-PLSA-LDA as a scable LDA-type model
for topic modeling and link prediction.  Relational Topic Model (RTM) was proposed by \cite{chang2009relational}
to model link between documents as a binary random variable based on the content of the document.  They
do not consider the community information.  \cite{Rosen-Zvi:2004} propose the author-topic model to
simultaneously model the content of the topic and the interest of the author using a shared hyperparameter.  
\cite{McCallum:2007} propose the topic-author-recipient model to take into account the directionality
of the link between the documents, and models the ``who-cited-whom" information.
More recently, we have works of \cite{El-Arini:2013} which represents documents through `badges',
which are essentially descriptive terms from the users sharing the documents.  However, in our
method, we model the full authorship information as a matrix and perform collective matrix factorization.
We note that none of these methods that use the link structure or authorship information consider temporal
aspect for monitoring topics.
