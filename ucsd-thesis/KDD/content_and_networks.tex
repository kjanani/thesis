%!TEX root = paper.tex
In this section, we explain how we formulate and optimize the problem of topic discovery and evolution
using content and social context
information.  Henceforth, we refer to our method as \textbf{LTECS}, an acronym for Learning Topic
Evolution from Content
and Social media activity.  We begin with some notation.  We assume a constant flow of documents.
Let $\mathbf{X}^t$ be a $N_d^t \times N_f$ matrix at time $t$ of $N_d^t$ documents and $N_f$ 
textual features.
The complete data matrix $\mathbf{X}$ obtained by concatenating vertically the matrices $\mathbf{X}^t$ along the time steps is considered huge and 
and practically difficult to store and handle.
The simplest approach to topic detection consists of directly learning from the global matrix $\mathbf{X}$.
However, in the real world, we are observing evolving topics and trends \cite{McCallum:2007}.
Hence, using much older data to estimate current trends may lead to wrong inference. 
Another typical strategy, consists of directly learning topics from the current batch of data while 
ignoring the trend history.  One is therefore faced with the tradeoff between past and present observations.
While recent approaches modeling topic evolution do address this tradeoff \cite{Vaca:2014, AlSumait:2008}, they rely
only on the content of the documents as their primary mode of input.
In order to consider other modalities as well (e.g. the social context associated to user activities), we introduce 
in the remainder of the section a multimodal approach to model topic evolution.
For the social context input, we have associated to each document in $\mathbf{X}^t$,
a set of users who are interested in these documents.  Let
$\mathbf{U}^t$ be a $N_d^t \times N_u$ matrix at time $t$ of $N_d^t$ documents and $N_u$ users.  Here,
$N_u$ is the total number of users in the social network.  In particular, we have $\mathbf{U}^t_{ij} = 1$ if
document-$i$ has been mentioned by user-$j$, and it is $0$ otherwise.
\vspace{-0.1cm}
\subsection{The Objective Function}
Our aim is to discover topics using both $\mathbf{X}^t$ and $\mathbf{U}^t$.  We will start with
the traditional objective function for NMF and build on it.
The goal of non-negative matrix factorization is to decompose documents in terms of the underlying
latent topics.  Let us fix the number of topics to be $k$.  We would like to decompose $\mathbf{X}^t$
so that:
\begin{equation}
	\mathbf{X}^t \approx \mathbf{W}^t\mathbf{H}^t. \label{eq:topic_decomp}
\end{equation}
Here, the $\mathbf{H}^t$ is a $k \times N_f$ topic matrix.  Each row in $\mathbf{H}^t$ represents
an underlying latent topic. If the encoding features of $\mathbf{X}^t$ are the words themselves, then each entry
in $\mathbf{H}^t$ represents how frequently a particular word appears in a topic.  The $\mathbf{W}^t$ matrix
represents how each document is decomposed in terms of the topics found in $\mathbf{H}^t$.  It \emph{explains}
each document in terms of the topics discovered in the $\mathbf{H}^t$ matrix.

For each document, in addition to the textual features, we have information about which users
are interested in these documents.  Just as in Equation \ref{eq:topic_decomp}, where we decomposed
each document in terms of the latent topics, we can think of decomposing the documents in terms of the
latent communities found in the social network.  That is, we have:
\begin{equation}
	\mathbf{U}^t \approx \mathbf{W}^t\mathbf{G}^t. \label{eq:comm_decomp}
\end{equation}
The key assumption in our formulation is that we have a common decomposition matrix $\mathbf{W}^t$ for
both equations \ref{eq:topic_decomp} and \ref{eq:comm_decomp}.  Our assumption is that a 
particular community of users will be dedicated to a particular topic.  Hence, we should be able to
decompose a document in terms of its topic \emph{or} in terms of its communities in the same way. An article
about Kim Kardashian can be thought of being decomposed as $90\%$ showbiz and $10\%$ spread across the other
topics.  Our postulation is that, there is a community of users who show keen interest in showbiz news, perhaps
a community in teenage demographics.  Hence, the same document can be equivalently decomposed in terms of 
the community as $90\%$ community interested in showbiz and $10\%$ spread across the other communities.
Equations \ref{eq:topic_decomp} and \ref{eq:comm_decomp} form the backbone of the two different parts
(namely the topic and the community part) to our objective function.  The way through which we connect
the two modalities is via the $\mathbf{W}^t$ matrix, making it common to both decompositions. 
This method is traditionally referred as \emph{collective factorization}  \cite{singh:2008}, and consists of sharing one common variable across different modalities.
The same principles have also been applied in deep learning 
(by sharing a common hidden layer across different modalities)~\cite{ngiam2011multimodal}, 
and in probabilistic modeling 
(by conditioning different observed modalities on a common hidden random variable)~\cite{Blei:2003}. 

Since we also wish to model topics' evolution over time, we make use of the topics that were
discovered in the previous time steps to help in better identifying topics
in the current influx of documents.  We decompose the current influx of documents using the topics discovered
in the previous time step as follows:
\begin{equation}
\mathbf{X}^t \approx \mathbf{W}^t\mathbf{M}_T^t\mathbf{H}^{t-1}. \label{eq:prev_topic_decomp}
\end{equation}
Here, $\mathbf{H}^{t-1}$ is a matrix of topics discovered in the previous time step.  The product
$\mathbf{M}_T^t\mathbf{H}^{t-1}$ can be thought of explaining the current topics $\mathbf{H}^t$ 
as a linear combination of the previous topics.  $\mathbf{M}_T^t$ is the \emph{topic evolution} matrix.
An $\mathbf{M}_T^{t}$ matrix close to identity (or a permuation of it) tells us that the topics have not changed 
much from the previous to current time step.  We delve into analyzing this matrix, and hence
the stabiltiy of topics (and communities) in future sections.

We also add a component of monitoring communities over time.  Similar to Equation \ref{eq:prev_topic_decomp},
we model the current set of documents with respect to the previous communities as follows:
\begin{equation}
\mathbf{X}^t \approx \mathbf{W}^t\mathbf{M}_C^t\mathbf{G}^{t-1}, \label{eq:prev_comm_decomp}
\end{equation}
where $\mathbf{M}_C^t$ is the community evolution matrix.

The crux of our loss function is formed by putting together Equations \ref{eq:topic_decomp} through
\ref{eq:prev_comm_decomp}.  Our variables are $\mathbf{W}^t$, $\mathbf{H}^t$, $\mathbf{G}^t$, 
$\mathbf{M}_T^t$ and $\mathbf{M}_C^t$.  The optimization is performed one time step after another.
Hence, $\mathbf{H}^{t-1}$ and $\mathbf{G}^{t-1}$ are \emph{known} to us by time $t$.  We decompose
our loss function into the following components,
\begin{equation}
L = \mu L_T + (1-\mu) L_C + \text{R}, \label{eq:loss_function}
\end{equation}
where $L_T$ and $L_C$ are the topic and community parts of the objective function and R encompasses the regularization terms.
We impose $l_1$ regularization on $\mathbf{W}^t$, $\mathbf{H}^t$, $\mathbf{G}^t$ and both the evolution matrices
$\mathbf{M}_T^t$ and $\mathbf{M}_C^t$ to promote sparsity.  
In order to drive the loss function more towards either topic modality or the community modality of the objective,
we use a parameter $\mu \in \big[0,1\big]$.  $\mu = 0$ places full weight on the community part and $\mu = 1$ places
full weight on the topic part.

The topic part and the community part of the
objective, and the regularization terms can be written as:
\begin{align} 
L_T &= ||\mathbf{X}^t - \mathbf{W}^t\mathbf{H}^t||_F^2 +  ||\mathbf{X}^t - \mathbf{W}^t\mathbf{M}_T^t\mathbf{H}^{t-1}||_F^2,  \\
L_C &= ||\mathbf{U}^t - \W\mathbf{G}^t||_F^2 +  ||\mathbf{U}^t - \mathbf{W}^t\mathbf{M}_C^t\mathbf{G}^{t-1}||_F^2,  \\
\text{R} &= \alpha(||\mathbf{W^t}||_1 + ||\mathbf{H}^t||_1 + ||\mathbf{G}^t||_1 + ||\mathbf{M}_T^t||_1 \nonumber \\ 
		& \qquad + ||\mathbf{M}_C^t||_1) + \lambda(||\mathbf{M}_T^t - I||_F^2 + ||\mathbf{M}_C^t - I||_F^2).
\end{align}
We add a term $\lambda||\mathbf{M}^t - \mathbf{I}||_F^2$
which, depending on the value of $\lambda \in \{0,\infty\}$ controls how much importance is placed on the past and the
present. A large value of $\lambda$ places much weight on the past and vice versa.  The role of parameters $\lambda$ and
$\mu$ are analyzed in detail in Section \ref{sec:experiments}.

\subsection{The Optimization}
We minimize the loss function $L$ as shown below:
\begin{equation}
\{\mathbf{W}^t_*, \mathbf{H}^t_*, \mathbf{G}^t_*, \mathbf{M}_{T,*}^t, \mathbf{M}_{C,*}^t\} =  \textbf{argmin } \underset{\mathbf{W}^t,\mathbf{H}^t,\mathbf{G}^t,\mathbf{M}_T^t, \mathbf{M}_C^t}  L. \label{eq:optimization}
\end{equation}
Note the variables with respect to which we optimize $L$.  Of these variables, the one that is most useful for evaluation
purposes is the matrix $\HH$.  This is a matrix of word distributions for each topic.  We compare the top-$10$ words from
each topic in $\HH$ to the top-$10$ obtained from the groundtruth.  More details about groundtruth and evaluation
are provided in Section \ref{sec:experiments}.

The optmization problem in Equation \ref{eq:optimization} is not convex in all the parameters simultaneously.  
We use multiplicative updates as
in \cite{lee_1999}.  For the loss function in Equation \ref{eq:optimization}, we derive the gradients with respect to each
variable as:
\begin{eqnarray}
\begin{split}
& \bigtriangledown_{\W}L = \W(\HH{\HH}^{T} + \G{\G}^{T} \\
		&\quad {\MT}^{T}{\Htt}^{T}\Htt\MT + {\MC}^{T}{\Gtt}^{T}\Gtt\MC) \\
	&\quad - (\X{\HH}^{T}+\X{\Htt}^{T}{\MT}^{T} + \U{\G}^{T}\\
	&\quad+\U{\Gtt}^{T}{\MC}^{T} - \alpha\e{\e}^{T}) , \label{eq:gradient_W}
\end{split} \\
\bigtriangledown_{\HH}L = {\W}^T\W\HH - ({\W}^T\X - \alpha\e{\e}^T), \\
\bigtriangledown_{\G}L = {\W}^T\W\G - ({\W}^T\U - \alpha\e{\e}^T), \\
\begin{split}
\bigtriangledown_{\MT}L = &  (\HH{\HH}^{T}){\MT}^{T}({\W}^{T}\W) + \lambda {\MT}^{T} \\
						&\quad- (\HH{\X}^{T}\W+\lambda\I - \alpha\e{\e}^{T}), \\
\end{split} \\
\begin{split}
\bigtriangledown_{\MC}L = &  (\G{\G}^{T}){\MT}^{T}({\W}^{T}\W) + \lambda {\MC}^{T} \\
						&\quad  - (\G{\U}^{T}\W+\lambda\I - \alpha\e{\e}^{T}), \\ \label{eq:gradient_MC}
\end{split}
\end{eqnarray}
where $\e = [1,1,\ldots,1].$
From the Karush Kuhn Tucker first order conditions, we have the primal feasibility as:
\vspace{-0.2cm}
\begin{equation}
\W \geq \0, \HH \geq \0, \G \geq \0, \MT \geq \0 \text{ and } \MC \geq \0,
\label{eq:primal_feasibility}
\end{equation}
the stationarity condition as $L(\W,\HH,\G,\MT,\MC) = 0$, at the minimizers, ${\W}^*, {\HH}^*, {\G}^*, {\MT}^*
{\MC}^*$, and the complementary slackness:
\begin{equation}
\begin{split}
\bigtriangledown_{\G}L \odot \G & = \mathbf{0} \text{,  } \bigtriangledown_{\HH}L \odot \HH = \mathbf{0}, \\
\bigtriangledown_{\MC}L \odot \MC &\quad = \mathbf{0} \text{,  } \bigtriangledown_{\MT}L \odot \MT = \mathbf{0}, \\
\bigtriangledown_{\W}L \odot \W & \quad = \mathbf{0}.
\label{eq:KKT}
\end{split}
\end{equation}
The update equations are derived by substituting the gradients (Equations \ref{eq:gradient_W} - \ref{eq:gradient_MC}) 
in the first order conditions (Equation \ref{eq:KKT}) as below:
\begin{align}
\begin{split}
&\W \gets \W \odot \frac{N}{D}, \text{ where } \label{eq:W_update}\\
N &= (\X{\HH}^{T}+\X{\Htt}^{T}{\MT}^{T} + \U{\G}^{T}+\U{\Gtt}^{T}{\MC}^{T} \\
	&- 2\alpha\e{\e}^{T}),\\
D &= \W(\HH{\HH}^{T} + \G{\G}^{T} + {\MT}^{T}{\Htt}^{T}\Htt\MT \\
	& + {\MC}^{T}{\Gtt}^{T}\Gtt\MC), \\ 
\end{split} \\
\HH \gets &\quad \HH \odot \frac{({\W}^{T}\X-\alpha\e{\e}^{T})}{{\W}^{T}\W\HH}, \label{eq:H_update}\\
\G \gets &\quad \G \odot \frac{({\W}^{T}\U-\alpha\e{\e}^{T})}{{\W}^{T}\W\G}, \label{eq:G_update}\\
\MT \gets &\quad \MT \odot \frac{(\Htt{\X}^{T}\W+\lambda\I - \alpha)}{(\Htt{\Htt}^{T}){\MT}^{T}({\W}^{T}\W) + \lambda {\MT}^{T}}, \label{eq:MT_update}\\
\MC \gets &\quad \MC \odot \frac{(\Gtt{\U}^{T}\W+\lambda\I - \alpha)}{(\Gtt{\Gtt}^{T}){\MC}^{T}({\W}^{T}\W) + \lambda {\MC}^{T}} \label{eq:MC_update}.
\end{align}

 \textbf{Theorem 1} \emph{The loss function $L$ in Equation  (\ref{eq:loss_function}) is non increasing under the update rules in Equations (\ref{eq:W_update}),  (\ref{eq:H_update}), (\ref{eq:G_update}), (\ref{eq:MT_update}), and (\ref{eq:MC_update}).  The loss function $L$ is invariant under these updates if and only if  $\HH$, $\G$,  $\MT$ and $\MC$ are at a stationary point of the function.} The proof for update rules on $\HH$ and $\G$ comes directly from \cite{Lee:Nature}.  For the update rules $\MT$ and $\MC$ a related formal proof is given in \cite{Vaca:2014}.  Finally, the proof for update rules on $\W$ follows from \cite{Lee:Nature}. Due to the lack of space, its proof will be provided in an extended version of the paper.
\vspace{-0.1cm}
