%!TEX root = paper.tex
%

% What am I trying to say?
% 1.  In general, there is a lot of information around these days.
% 2.  We need sophisticated techniques which can pick out the most useful ones, and those
% which can detect trends e.t.c.
% 3.  Under the umbrella of topic discovery this area has always gained a lot of interest.
% 4.  With the advent of Web 2.0 the problem has been made even more challenging because each
% of us is a source of information.

%The field of topic discovery and evolution (TDE) has always garnered plenty of interest from the research community.
Topic discovery has been a well studied
area of research since the 90s.  In recent times, this area has gained renewed interest with the
advent of social media \cite{Sayyadi:2009, Becker:2009}.  Social media has completely changed the dynamics of how we operate as a
society, and has given each of us the power of being able to constantly produce and
share content with the rest of the world.  This presents several new challenges to the field of
topic discovery and evolution (TDE).  Firstly, the content is constantly evolving.  Hence, any topic discovery algorithm
needs to keep with the ever changing nature of the content.  And secondly, since each one of us has the power
to produce and share content, the vocabulary used to describe a particular event can be quite varied.  In addition to this,
it has also resulted in an enormous explosion in the size of data making it more challenging to identify
which posts are indeed the important ones.
Hence algorithms need to be robust to such irregularities in the textual content, and should identify
the most important and relevant information from a large corpus.

Many classical TDE algorithms aim to detect the underlying latent topics from
the textual content of the data alone.  They are blind to the social context that comes along with the
text.  For example, Twitter contains information about the user, geographical
location, time of post, etc, which could be very useful information as
it gives a context to the textual
content.  We could incorporate such context information in our learning process
to learn better topics.
%In today's scenario of topic discovery, textual
%content comes with a social context which could be leveraged to detect better topics.
What we propose in this paper is to use the information about communities present
in social media in addition to textual content to discover and track topics. 
We propose to use the information about the authorship of posts, shares and comments to
detect communities
of users.  By definition, members of the same community will exhibit
common interests in sharing and posting information about a particular topic.  The idea here is to
leverage this information (as side information) to retrieve more accurate topics.

%Topic discovery and tracking (TDT) is about discovering trends present in textual data.  

%Why do we think that, by using social communities, it will help in detecting better
%topics?  The answer is that 
We hypothesize that some topics are particularly difficult to discover using textual
content alone either because the text present in the topic uses a widely varying vocabulary or that the text could 
be very volatile and could change in a very short period of time.  An example is: \emph{celebrity gossip}.  
For such a topic, the content varies
from one celebrity to another and hence can have a widely varying vocabulary, or can take unexpected turns 
(celebrity break-ups, pregnancies, etc.) and thus be volatile.  
But, for the same topic, we could
have a very dedicated community of users who constantly share posts and comments about the topic.
In this case of \emph{celebrity gossip}, it perhaps finds its niche audience largely in the teenage demographics.
Hence, leveraging the presence of community could be useful in discovering such
difficult topics.  We will call such topics which have a volatile and constantly
changing text, but a farily dedicated community of users following the topic to
be \emph{community stable topics}.  There could be some other topics which have a focused textual
content that does not change much over time.  
For these seemingly ``easier" topics (which we will call \emph{content stable topics}), perhaps using the community of interested users
as side information may not 
particularly improve topic discovery\footnote{In Section \ref{sec:data_set_description}, we will
explain in detail how we obtain the content stable topics, community stable topics
etc.}.  One cannot say with surity (yet).  This is precisely the kind
of question we provide an answer to in this paper.
%For the rest of the paper, we name such topics as \emph{community stable}
%topics meaning they have a stable community of interested followers.  
%There maybe certain other topics which could easily be discovered
%using the textual content alone.  We consider those topics as well to see if using community
%as side information at all affects the topic discovery performance.  We name such topics as
%\emph{content stable} topics as they have very strong textual content decribing the topic.
%We also consider a third category called \emph{mixed stable} topics to which refer to those
%topics which have both stable content and community information about them.

To summarize, our aim in this paper is to study in detail if and when the presence of social
context information is useful in discovering and monitoring topics. 
There are works in literature which have combined both content and link information, but
the natural time-based evolving and changing aspects of both have not been considered. 
Other works track content along time, but do not accommodate
for both content and link information. 
To the best of our knowledge, we are the first to propose an approach that exploits simultaneously
 both the content and the social context  in a unified framework for modeling topic evolution.  
We build on the non-negative matrix factorization (NMF) objective, with
different terms modeling the content and social context aspects of our problem.  Within each modality, we model their
temporal evolution to learn from the past as well.  To learn both modalities at the same time, and 
to exploit their correlation, 
we rely on a \emph{collective factorization} approach as introduced in \cite{singh:2008}. 
In our context, it consists of sharing at each time step, a common variable
representing both the topic and the community distributions 
during the learning process (for more details, see Section \ref{sec:content_and_networks}).

We perform experiments on a publicly available large scale dataset of news articles
\cite{YatesTrumper:2013}, and use the Twitter social networking platform as the
source of social context for the news articles.
We study the effects of social activity as side information when modeling topic evolution.
%\inote{Amin: We need to update that part after results on Tumblr are available}.  
In particular, our focus will be on the following research questions. 
\begin{enumerate}
\item Does the presence of community as side information help in discovering those topics which have a strongly
focused textual content (i.e. \emph{content stable} topics)?  One may generally
not expect the presence of community to help in this case, since the strong textual content probably
suffices to discover the topics.  However, through our experiments, we indeed see improvements in performance
in some cases. \label{q:q2}
\item Does the presence of community help in discovering those topics which do not have stable textual
content, but have stable communities of members interested in them (i.e. \emph{community stable} topics)?
In this case, we observe remarkable improvements in performance when we use the community information, as
opposed to only using textual content. \label{q:q3}
\item Does the presence of community help in discovering topics which have both a stable textual
content and a stable community (i.e. \emph{mixed stable} topics)?  We observe improvements in performance
in this case as well. \label{q:q4}
\item How does our algorithm compare to existing state of the art which model topic evolution
and those which model document link structure for topic discovery?  In particular, we compare
our algorithm to Link-PLSA-LDA \cite{AlSumait:2008}, a generative model that incorporates context and content (but no tracking); 
Collective Matrix Factorization \cite{Saveski:2014, Ding:2014}, an NMF-based model that incorporates context
and content (but no tracking); Joint Past Present Decomposition \cite{Vaca:2014}, an NMF based topic tracking model, 
and Online-LDA, \cite{Nallapati:2008}, a generative topic tracking model.  
We indeed outperform the state-of-the-art in several instances. \label{q:q5}
\item To what extent can our algorithm 
learn the kind of topics and communities that are at hand?  That is, given
an input stream of documents, how well can the algorithm figure out whether the topics that have been detected
are content stable, community stable or mixed stable? \label{q:q1}
\end{enumerate}

The rest of the paper is organized as follows.  Section \ref{sec:comparison_to_previous_work} compares
our work to existing work highlighting its novelties and differences over them.  
Section \ref{sec:content_and_networks} explains our model loss function in detail, and how
we optimize it.  Section \ref{sec:data_set_description} describes the dataset we used in our experiments.
Section \ref{sec:experiments} provides a detailed explanation of the experiments and results,
and we finish with the conclusion in Section \ref{sec:conclusion}.
\vspace{-0.2cm}
