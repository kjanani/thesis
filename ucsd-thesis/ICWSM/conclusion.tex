\section{Conclusion}
\label{sec:conclusion}
We present a timeline of events during 
the 2014 Ebola outbreak as manifested on Twitter. 
The final timeline reveals
three main categories of events based on their
longevity:  \emph{long-term}, \emph{medium-term}
and \emph{short-term} events.  Some unique characteristics
about the categories emerged upon further analysis.
One, \emph{long-term} events tend to be those with repercussions
that might possibly put the health of the public at risk
(e.g. suspected new cases in the US), and hence cause
anxiety in the general public.  
Second, any event that has a positive connotation (e.g. philanthropic
donations to help in the Ebola effort) tends to be short-lived and
quickly disappears from the timeline.   
Third, a substantial number of the  short-lived events that emerge from the data set are
memes and jokes related to Ebola.  Fourth,  some episodes
of public health importance (e.g. new cases) tend to generate
short-term events if these occur outside the US.
Insights gleaned from these patterns could be of use to public 
health organizations like the United States Centers for 
Disease Control and Prevention in future outbreaks 
for supporting situational awareness, 
monitoring the strength and duration of public concerns, 
and identifying potential opportunities for health education.
