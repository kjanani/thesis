\section{Introduction}
Ebola virus disease is a communicable disease characterized by severe
symptoms (e.g. nausea, vomiting, haemorrhaging) and a very high
fatality rate \cite{WHO-Ebola-Response-Team:2014aa}.   The 2014 Ebola
outbreak --- the largest outbreak of Ebola in history --- began in
Guinea around January 2014 and rapidly spread to neighboring West
African countries (Sierra Leone, Liberia), with the World Health
Organization classifying the outbreak as a \emph{Public Health
  Emergency of International Concern} in August of that
year \cite{Weyer:2015aa}.  As of December 31st 2015, it is estimated
that there have been almost 30,000 cases associated with the outbreak,
with over 11,000 deaths\footnote{www.webcitation.org/6ePMzFc5a}.  Only 7 cases occurred outside Africa (4 in
the United States, and 1 each in Italy, the United Kingdom, and
Spain), resulting in 1 fatality in the United States.  Despite the
very low disease prevalence in the United States, public concern
regarding Ebola risk remained elevated throughout the
outbreak \cite{Towers:2015aa}.

Publicly available social media services, in particular micro-blog platforms like Twitter, are
increasingly recognized as a valuable data source for understanding
public attitudes and
opinions towards health issues, especially when combined with
computational approaches like Natural
Language Processing and Machine Learning \cite{Dredze:2012qy}.     Public attitudes towards vaccination
\cite{Salathe:2011aa}, novel tobacco products \cite{Myslin:2013aa},
illegal drugs \cite{Krauss:2015aa}, and eating disorders \cite{DBLP:conf/ehealth/Choudhury15} have all been investigated
using a combination of social media data and computational techniques.   

% Twitter, in conjunction with
% computational methods, has been used with success in
% understanding public attitudes towards vaccination
% \cite{Salathe:2011aa}, novel tobacco products \cite{Myslin:2013aa},
% and illegal drugs \cite{Krauss:2015aa}.    

In this work, 
%which builds on previous work investigating
%Ebola rumors [reference left-out for double blind review] %\cite{DBLP:journals/corr/KalyanamVDCL15},  
we used a corpus of 10.5 million Ebola-related tweets
generated in the United States during October 2014 --- the height of
the outbreak ---
%\textcolor{blue}{JANANI:  ML learning-based
%  topic modeling methods} 
to create \emph{topical timelines} that reveal 
topic progression during the outbreak by using a novel topic modeling
algorithm that explicitly models the evolution of topics
as emerging, fading, and changing.
In order to develop the
timelines, we followed a three step approach: (1) segment the data into
daily blocks, (2) apply our topic modeling algorithm to each daily
block, and (3) map and merge daily topics to create a topical
timeline (i.e. events).

An analysis of our empirically-derived Ebola timeline showed that, in terms of
their temporal duration,  identified events fell into three main groups:  \emph{long-term} events which
lasted at least 5 days,  \emph{medium-term} events which lasted between
2 and 4 days, and \emph{short-term} events which lasted for 1 day.
Generally,  long-term, enduring events tend to be those with repercussions
that might possibly put the health of the public at risk
(e.g. suspected new cases in the United States), and hence cause
anxiety in the general public.   We also observed
that any event which has a positive connotation (e.g. philanthropic
donations to help in the Ebola effort) tends to be either a
short-term or medium-term event.   Further, we discovered that a
substantial number of the  short-lived events (i.e. short-term and
medium-term events) that emerged from the data set are
memes and jokes related to Ebola.   Finally, we noted that some episodes
of public health importance (e.g. new cases) tend to generate
short-term events if these occur outside the United States,
indicating that overseas, non-US events are of limited interest to US-based
Twitter users.


 Insights gleaned from this 
Twitter-based event tracking method could be utilized by
both international public health organizations (e.g. \emph{World Health
Organization}, \emph{European Centre for Disease Prevention and Control}) and national
public health entities (e.g. United States \emph{Centers for Disease
  Control and Prevention}, \emph{Public Health Agency of Canada})  to
monitor public opinion in future outbreaks with the goal of  supporting situational 
awareness, assessing the strength and duration of public concerns, 
and identifying potential opportunities for health education.

The paper is structured as follows.  Section \ref{sec:related} describes related work;
Section \ref{sec:data_and_model} describes the data and its attributes; Section \ref{sec:timeline_generation} describes
the event timeline generation methodology, and also introduces a novel topic model
which explicitly incorporates evolving content into its framework; Section \ref{sec:comparison}
presents a quantitative evaluation of the topic model(s) used in this work;
Section \ref{sec:timeline} presents the findings and inferences of the timeline created
from Section \ref{sec:timeline_generation}.  We end with a conclusion in Section \ref{sec:conclusion}.

