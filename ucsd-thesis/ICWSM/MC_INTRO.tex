\def\year{2015}
%File: formatting-instruction.tex
\documentclass[letterpaper]{article}
\usepackage{AuthorKit/aaai}
\usepackage{times}
\usepackage{url}
\usepackage{pbox}
\usepackage{helvet}
\usepackage{courier}
\usepackage{color}
\usepackage{amsmath}
\usepackage{amssymb}
\usepackage{amsfonts}
\usepackage[square,sort,comma,numbers]{natbib}
\usepackage{graphicx}
\usepackage{setspace,calc}
\usepackage{supertabular}
\usepackage{multirow}
\usepackage{xcolor}
\frenchspacing
\setlength{\pdfpagewidth}{8.5in}
\setlength{\pdfpageheight}{11in}
%\pdfinfo{
%/Title (Write-up)
%\title{Write-up}
%/Author (Put All Your Authors Here, Separated by Commas)}
\title{Using Twitter and topic tracking to monitor public reactions
  towards  the 2014 Ebola outbreak}
\setcounter{secnumdepth}{2}  
 \begin{document}
\maketitle

\begin{abstract}
\begin{quote}
The 2014 Ebola outbreak --- the largest in history --- involved almost 
30,000 cases and over 11,000 fatalities.  
Only four of these cases (and one fatality) occurred in the 
United States, yet US public concern regarding Ebola risks, 
as manifested in both social media and traditional news sources, 
remained very high.

In this work, we use a corpus of 10.5 million English 
language Ebola-related tweets derived from Twitter users in 
the United States in October 2014 --- the height of the 
Ebola crisis --- to investigate Ebola-related topics and topic 
development in social media.  Using the results of two machine 
learning algorithms, one of which we propose in this work, 
we produce a timeline of Ebola-related events as they emerged 
during October 2014, discovering four distinct patterns.   First, long-term, enduring topics tend to be those with repercussions
that might possible put the health of the public at risk
(e.g. suspected new cases in the United States), and hence cause
anxiety in the general public.  Second, we observed
that any topic that has a positive connotation (e.g. philanthropic
donations to help in the Ebola effort) tends to be short-lived and
quickly disappears from the timeline.   Third, we discovered that a
substantial number of the  short-lived topics that emerge from the data set  are
memes and jokes related to Ebola.  Fourth,  we noted that some events
of public health importance (e.g. new cases) tend to generate
short-term topics if these events occur outside the United States.

 Insights gleaned from this 
Twitter-based event tracking method could be of use to
public health organizations like the United States \emph{Centers 
for Disease Control \& Prevention} in future outbreaks for supporting situational 
awareness, monitoring the strength \& duration of public concerns, 
and identifying potential opportunities for health education.


% First, any event or topic with potential
% repercussions that could put the health of
% the US public at risk seemed to engage Twitter users for much
% longer. Second, generally, topics which had a positive note
% to them were not long terms topics. Third, generally
% any topic which is about a nation other than the United
% States lasted only for 1 day. Fourth, memes/jokes/rumours
% lasted anywhere from 1 to 4 days. Insights gleaned from this 
% Twitter-based event tracking method could be of use to
% public health organizations like the United States Centers 
% for Disease Control in future outbreaks for supporting situational 
% awareness, monitoring the strength \& duration of public concerns, 
% and identifying potential opportunities for health education.
\end{quote}
\end{abstract}  

\section{Introduction \& Motivation}
Ebola virus disease is a communicable disease characterised by severe
symptoms (e.g. nausea, vomiting, haemorrhaging) and a very high
fatality rate\cite{WHO-Ebola-Response-Team:2014aa}.   The 2014 Ebola
outbreak --- the largest outbreak of Ebola in history --- began in
Guinea around January 2014 and rapidly spread to neighbouring West
African countries (Sierra Leone, Liberia), with the World Health
Organization classifying the outbreak as a \emph{Public Health
  Emergency of International Concern} in August of that
year\cite{Weyer:2015aa}.  As of December 31st 2015, it is estimated
that there have been almost 30,000 cases associated with the outbreak,
with over 11,000 deaths\footnote{www.webcitation.org/6ePMzFc5a}.  Only 7 cases occurred outside Africa (4 in
the United States, and 1 each in Italy, the United Kingdom, and
Spain), resulting in 1 fatality in the United States.  Despite the
very low disease prevalence in the United States, public concern
regarding Ebola risk remained elevated throughout the
outbreak\cite{Towers:2015aa}.

Publicly available social media services, in particular micro-blog platforms like Twitter, are
increasingly recognised as a valuable data source for understanding
public attitudes and
opinions towards health issues, especially when combined with
computational approaches like Natural
Language Processing (NLP) and Machine Learning (ML)\cite{Dredze:2012qy}.     Public attitudes towards vaccination
\cite{Salathe:2011aa}, novel tobacco products \cite{Myslin:2013aa},
illegal drugs \cite{Krauss:2015aa}, and eating disorders\cite{DBLP:conf/ehealth/Choudhury15} have all been investigated
using a combination of social media data and computational techniques.   

% Twitter, in conjunction with
% computational methods, has been used with success in
% understanding public attitudes towards vaccination
% \cite{Salathe:2011aa}, novel tobacco products \cite{Myslin:2013aa},
% and illegal drugs \cite{Krauss:2015aa}.    

In this work, which builds on previous work investigating
Ebola rumours \cite{DBLP:journals/corr/KalyanamVDCL15},  we used a corpus of 10.5 million Ebola-related tweets
generated in the United States during October 2014 --- the height of
the outbreak ---  and then using \textcolor{blue}{JANANI:  ML learning-based
  topic modeling methods} created \emph{topical timelines} that reveal
topic progression during the outbreak.   In order to develop the
timelines, we followed a three step approach: (1) segment the data into
daily blocks, (2) apply our topic modeling algorithm to each daily
block, and (3) map and merge daily topics to create a topical
timeline. 

An analysis of our empirically-derived Ebola timeline showed that, in terms of
their temporal duration,  identified topics fell into three main groups:  \emph{long-term} topics which
lasted at least 5 days,  \emph{medium-term} topics which lasted between
2 and 4 days, and \emph{short-term} topics which lasted for 1 day.
Generally,  long-term, enduring topics tend to be those with repercussions
that might possible put the health of the public at risk
(e.g. suspected new cases in the United States), and hence cause
anxiety in the general public.   We also observed
that any topic which has a positive connotation (e.g. philanthropic
donations to help in the Ebola effort) tends to be either a
short-term or medium-term topic.   Further, we discovered that a
substantial number of the  short-lived topics (i.e. short-term and
medium-term topics) that emerged from the data set are
memes and jokes related to Ebola.   Finally, we noted that some events
of public health importance (e.g. new cases) tend to generate
short-term topics if these events occur outside the United States,
indicating that overseas, non-US events are of limited interest to US-based
Twitter users.


 Insights gleaned from this 
Twitter-based event tracking method could be utilised by
both international public health organizations (e.g. \emph{World Health
Organization}, \emph{European Centre for Disease Prevention and Control}) and national
public health entities (e.g. United States \emph{Centers for Disease
  Control and Prevention}, \emph{Public Health Agency of Canada})  to
monitor public opinion in future outbreaks with the goal of  supporting situational 
awareness, assessing the strength \& duration of public concerns, 
and identifying potential opportunities for health education.

\textcolor{blue}{JANANI: The paper is structured as follows.  Section 2 describes related work
.  Section 3....  Structure of the paper paragraph.}

 
\section{Background \& Related Work}
\subsection{Disease threats}
Infectious diseases continue to pose a serious threat to global
health.  Emerging zoonotic diseases (e.g. \emph{Middle East Respiratory
Syndrome}\footnote{www.webcitation.org/6eQV6BcML}), re-emerging
antibiotic resistant diseases
(e.g. \emph{tuberculosis}\footnote{www.webcitation.org/6eQVG0xMO}), and
bioterrorism threats
(e.g. \emph{pestis}\footnote{www.webcitation.org/6eQVKzjK1}) all pose serious
challenges to global health security, especially given increases in
the frequency and speed of international travel, and greater population
density \cite{Hartley:2010aa}. %However, outbreak detection and
%surveillance is just part of the public health response, equally important


\subsection{Digital Disease Detection \& Ebola}
The application of computational methods to social media, news
reports, and 
consumer-generated non-clinical ``big data'' is a rapidly growing area
of research, variously referred to as \emph{digital disease detection}
\cite{Brownstein:2009aa},
\emph{digital epidemiology} \cite{Salathe:2012aa}, and \emph{infodemiology}\cite{Eysenbach:2009aa}, with a focus
both on tracking disease incidence and prevalence (e.g. Google Flu
Trends \cite{Cook:2011aa})
and on the early identification of disease outbreaks (e.g. Healthmap \cite{Freifeld:2008aa},
Biocaster \cite{Collier:2008aa}).   There is now substantial evidence
that Twitter is a valuable resource for some infectious disease
monitoring applications, especially for influenza
\cite{Signorini:2011aa,Collier:2011aa}.
However, 
disease tracking and outbreak detection is just part of the public
health response to an outbreak \cite{Oyeyemi:2014aa}: understanding the strength and duration of public concerns, 
and identifying potential opportunities for health education are
vital, especially in situations where a disease is poorly
understood\cite{SteelFisher:2015aa} by the general public.  






\subsection{Event detection using topic modeling}
\textcolor{blue}{JANANI:  Details of Event detection and topic modeling here} 



%\section{Discussion}



\bibliographystyle{plain}
\bibliography{refs,MC_REFS}


\end{document}
