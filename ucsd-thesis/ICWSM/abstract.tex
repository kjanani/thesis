\begin{abstract}
\begin{quote}

The 2014 Ebola outbreak --- the largest in history --- involved almost 
30,000 cases and over 11,000 fatalities.  
Only four of these cases (and one fatality) occurred in the 
United States, yet US public concern regarding Ebola risks, 
as manifested in both social media and traditional news sources, 
remained very high.

In this work, we use a corpus of 10.5 million English 
language Ebola-related tweets derived from Twitter users in 
the US in October 2014 --- the height of the 
Ebola crisis --- to investigate Ebola-related events and event
development in social media.  Using the results of two machine 
learning algorithms, one of which we propose in this work, 
we produce a timeline of Ebola-related events as they emerged 
during October 2014, discovering four distinct patterns.   First, long-term, enduring events tend to be those with repercussions
that might possibly put the health of the public at risk
(e.g. suspected new cases in the US), and hence cause
anxiety in the general public.  Second, we observed
that any event that has a positive connotation (e.g. philanthropic
donations to help in the Ebola effort) tends to be short-lived and
quickly disappears from the timeline.   Third, we discovered that a
substantial number of the  short-lived events that emerge from the data set are
memes and jokes related to Ebola.  Fourth, we noted that events of 
public health importance (e.g. new cases) that occur 
\emph{outside} the United States typically generate short term events.

Insights gleaned from this 
Twitter-based event tracking method could be of use to
public health organizations like the US \emph{Centers 
for Disease Control \& Prevention} in future outbreaks for supporting situational 
awareness, monitoring the strength \& duration of public concerns, 
and identifying potential opportunities for health education.


% First, any event or topic with potential
% repercussions that could put the health of
% the US public at risk seemed to engage Twitter users for much
% longer. Second, generally, topics which had a positive note
% to them were not long terms topics. Third, generally
% any topic which is about a nation other than the United
% States lasted only for 1 day. Fourth, memes/jokes/rumors
% lasted anywhere from 1 to 4 days. Insights gleaned from this 
% Twitter-based event tracking method could be of use to
% public health organizations like the United States Centers 
% for Disease Control in future outbreaks for supporting situational 
% awareness, monitoring the strength \& duration of public concerns, 
% and identifying potential opportunities for health education.
%
%
%The 2014 Ebola outbreak --- the largest in history --- involved almost 
%30,000 cases and over 11,000 fatalities.  
%Only four of these cases (and one fatality) occurred in the 
%United States, yet US public concern regarding Ebola risks, 
%as manifested in both social media and traditional news sources, 
%remained very high.
%
%In this work, we use a corpus of 10.5 million English 
%language Ebola-related tweets derived from Twitter users in 
%the United States in October 2014 --- the height of the 
%Ebola crisis --- to investigate Ebola-related topics and topic 
%development in social media.  Using the results of two machine 
%learning algorithms, one of which we propose in this work, 
%we produce a timeline of Ebola-related events as they emerged 
%during October 2014, discovering four distinct patterns.   
%First, any event or topic with potential
%repercussions that could put the health of
%the US public at risk seemed to engage Twitter users for much
%longer. Second, generally, topics which had a positive note
%to them were not long terms topics. Third, generally
%any topic which is about a nation other than the United
%States lasted only for 1 day. Fourth, memes/jokes/rumors
%lasted anywhere from 1 to 4 days. Insights gleaned from this 
%Twitter-based event tracking method could be of use to
%public health organizations like the United States Centers 
%for Disease Control in future outbreaks for supporting situational 
%awareness, monitoring the strength \& duration of public concerns, 
%and identifying potential opportunities for health education.
%
%
%
%
%
%
%%Public health organizations like the Center for
%Disease Control (CDC) are very interested to know
%the kinds of topics that the general public engage in
%during an epidemic crisis. 
%Micro-blogging services like Twitter are used by the public, 
%organizations and news media to consume and disseminate information.  
%This was also the case following the first verified case of Ebola in the United 
%States in the fall of 2014. 
%An qualitative and quantitative understanding of the kind of topics that were discussed
%by the general public on Twitter during the Ebola crisis will be helpful for public 
%health organizations to be better prepared for the next epidemic crisis.
%In this work, we study a Twitter-Ebola dataset.  
%We downloaded 10.5 million tweets (Twitter messages) about Ebola which were 
%published on Twitter in the United Stated in October 2014.  Using the results of two machine learning
%algorithms, one of which we propose in this work, we produce a timeline of events
%as they emerged on Twitter during that time period.  We found the following
%patterns.  One, any event or topic whose repercussions could possibly,
%at large, put the health of the public at risk seem to engage the users
%for much longer.  Two, generally, topics which had a positive note to them
%were not long terms topics.  Three, generally any topic which is about a nation other than the
%United States lasted only for $1$ day.  Four, memes/jokes/rumors lasted anywhere
%from $1$ to $4$ days.
%Such inferences can be useful for organizations like the CDC to understand the
%behavior of the public, and address their concerns.
%
%
%
%During crisis situations, such as a disease outbreak, it is important for 
%public health organizations to track
%and monitor the general well being of the public.  In addition to performing close
%surveillance of symptomatic patients, organization like the Center for Disease
%Control (CDC) are also interested in addressing the concerns of the public.
%What will be of immense interest to organizations
%like the CDC is an insight into the kinds of topics and discussion that the general public
%engaged in during that time period. 
%
%Micro-blogging services like Twitter are used by both the public, 
%organizations and news media to consume and disseminate information.  
%This was also the case following the first verified case of Ebola in the United 
%States in the fall of 2014. 
%An qualitative and quantitative understanding of the topics that were discussed
%by the general public on Twitter during the Ebola crisis will be helpful for public 
%health organizations to be better prepared for the next epidemic crisis.
%
%In this work, we study a Twitter-Ebola dataset.  
%We downloaded 10.5 million tweets (Twitter messages) about Ebola which were 
%published on Twitter in the United Stated in October 2014.  Using the results of two machine learning
%algorithms, one of which we propose in this work, we produce a timeline of events
%as they emerged on Twitter during that time period.  We found some topics like
%concerns about airport screening and security lasted through out the time period of the dataset.
%We also found memes/jokes that lasted anywhere between $3$ to $6$ days.
%There were also some topics which lasted only a day.  These, for example, were
%about events in another country.
\end{quote}
\end{abstract}
