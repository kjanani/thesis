\section{Related Works}
\label{sec:related}
%\subsection{Digital Disease Detection \& Ebola}
The application of computational methods to social media, news
reports, and 
consumer-generated non-clinical ``big data'' is a rapidly growing area
of research, variously referred to as \emph{digital disease detection}
\cite{Brownstein:2009aa},
\emph{digital epidemiology} \cite{Salathe:2012aa}, and \emph{infodemiology} \cite{Eysenbach:2009aa}, with a focus
both on tracking disease incidence and prevalence (e.g. Google Flu
Trends \cite{Cook:2011aa})
and on the early identification of disease outbreaks (e.g. Healthmap \cite{Freifeld:2008aa},
Biocaster \cite{Collier:2008aa}).   There is now substantial evidence
that Twitter is a valuable resource for some infectious disease
monitoring applications, especially for influenza
\cite{Signorini:2011aa,Collier:2011aa}.
However, 
disease tracking and outbreak detection is just part of the public
health response to an outbreak \cite{Oyeyemi:2014aa}: understanding the strength and duration of public concerns, 
and identifying potential opportunities for health education are
vital, especially in situations where a disease is poorly
understood \cite{SteelFisher:2015aa} by the general public.  
%One major reason for this is that the content generated on
%micro blogging platforms like Twitter
%is highly infiltrated with noise (since it is not curated), hence rendering it difficult
%for computational/statistical approaches (like topic models) to automatically make inferences
%and summarize the content \cite{Rajani:2014, Mehrotra:2013, Hong:2010}.
Although advancements have been made in the area of detection and
identification of events from social media, these contributions
have been more methodological (step-wise procedures) than algorithmic \cite{weng2011event, li2012tedas, becker2011beyond, cataldi2010emerging}.  
While, at the time of an epidemic, what is really needed is a more automatic approach with minimum
human involvement to identify potential
intervention areas for official public health departments from a very high volume
of constantly evolving incoming data.
